\documentclass[11pt,a4paper]{article}
\usepackage[francais]{babel}
\usepackage[top=1cm, bottom=1.8cm, left=2cm, right=2cm, marginparwidth=0cm, marginparsep=0cm]{geometry}
\usepackage[utf8]{inputenc}

\usepackage{hyperref}
\usepackage{graphicx}
%\usepackage{listings}
\usepackage{keystroke}

% Commande \code
\usepackage{xcolor}
\definecolor{gris}{gray}{0.75}
\definecolor{grisleger}{gray}{0.95}
\definecolor{magenta}{HTML}{DD1144}
\newcommand{\code}[1]{\fcolorbox{gris}{grisleger}{{\color{magenta}{\texttt{#1}}}}}
% Commande bouton
\newcommand{\bouton}[1]{« \textit{#1} »}

\begin{document}
\title{Guide d'utilisation de Legion}
\author{Romain Hennuyer\\
	Lycée Jean Moulin - Roubaix}
\date{\today}
 
\maketitle
\tableofcontents
\pagebreak

% Installation
\section{Installation et Usage}
\subsection{Pré-requis}
Legion nécessite python 3 ainsi que notamment les librairies sqlite3, urllib, json, numpy, matplotlib, et xlrd.

\subsection{Installation}
Il suffit de cloner le dépôt public de Legion :\\
\code{git clone https://github.com/romainhk/Legion.git}

Ou de récupérer/décompresser l'archive à l'adresse :\\
\url{https://github.com/romainhk/Legion/archive/master.zip}

\subsection{Configuration}
Il ne reste plus qu'à personnaliser le fichier \texttt{config.cfg} avec :
\begin{itemize}
\item le port de connexion ;
\item les mot de passe administrateur et eps, encodés en SHA512 ; un outil en ligne permet de générer ça facilement : \url{http://caligatio.github.io/jsSHA/} ;
\item le nom de l'établissement ;
\item le nom des sections pour chaque filière.
\end{itemize}

\subsubsection{Choix du port}
N'importe quel numéro de port (au dessus de 1024) est assignable, tant que l'on vérifie qu'il n'est pas filtré par le par-feu.

À noter que sur le réseau académique, la plupart des ports sont bloqués en entrée et en sortie. Le BAIP peut faire remonter une demande d'ouverture de port au DAIP, comme par exemple le 49300 utilisé par pronote.

\subsection{Exécution du serveur}
En mode interactif :
\code{python3 legion.py}

Ou pour une utilisation en mode démon (à préférer) :

\code{nohup python3 legion.py \&}

Alors, le fichier legion.log est remplacé par nohup.out

\subsection{Se connecter sur la base}
Il suffit de rentrer l'adresse \url{http://IP_DU_SERVER:PORT} dans n'importe quel navigateur web (nécessite simplement javascript), puis de rentrer le mot de passe administrateur.

\subsection{Extinction du serveur}
En mode démon, le PID du processus serveur est noté dans le fichier de log au démarrage. Il suffit de faire un \code{kill -15 PID} pour que le serveur s'éteigne proprement.

En mode interactif, il suffit de faire un \keystroke{Ctrl} + \keystroke{C}.

\pagebreak
% Présentation des fonctionnalités
\section{Présentation des fonctionnalités}

Tout d'abord, deux profils de connexion sont disponibles :
\begin{itemize}
\item admin : avec tous les droits ;
\item eps : qui ne peut utiliser que la partie EPS et les stats concernant l'EPS.
\end{itemize}

\subsection{Onglets}

\includegraphics[width=0.7\textwidth,natwidth=603,natheight=118]{onglets.png}

\begin{itemize}
\item La partie \bouton{Liste} présente une liste triable et filtrable des élèves de l'établissement ;
\item La partie \bouton{Stats} offre différentes statistiques concernant la base élève ;
\item La partie \bouton{EPS} permet à un prof d'eps de saisir les activités et les notes des élèves ;
\item La partie \bouton{Attente} résume les importations qui se sont mal passées. Cela permet de détecter certains problèmes de déclaration dans la base SIECLE (pour correction en amont) ;
\item La partie \bouton{Options} permet d'importer des données, et d'affecter les classes à leur niveau et filière ;
\end{itemize}

Le bouton bleu permet de se déconnecter.

\subsection{Exportation}

\includegraphics[width=0.4\textwidth,natwidth=296,natheight=57]{exportation_csv.png}

Le bouton d'exportation, omniprésent, permet d'extraire les données présentées au format csv, exploitable par n'importe quel bon tableur.

\pagebreak
% Initialisation
\section{Initialisation d'une année scolaire}
\subsection{Récupération des données}
La base élève établissement (BEE) doit être récupéré dans SIECLE, accessible \href{http://pagriates.ac-lille.fr/}{sur le portail pagriates}.

En utilisant un compte ayant suffisamment de privilèges, aller dans « Scolarité du 2nd degré $>$ Base élève établissement (BEE) $>$ Mise à jour » :

\includegraphics[width=\textwidth,natwidth=995,natheight=608]{siecle1.png}

Dans « Exportations $>$ En XML », cliquer sur « Élèves sans adresses » :

\includegraphics[width=\textwidth,natwidth=995,natheight=710]{siecle2.png}

Cela permet de télécharger un fichier xml zippé contenant la base.

\subsection{Importation}
Décompresser le fichier récupéré ; on obtient le fichier \texttt{ElevesSansAdresses.xml}.

Se connecter à Legion et aller dans l'onglet \bouton{Options}. Cliquer sur \bouton{Parcourir}, sélectionner le fichier xml précédent puis cliquer sur \bouton{Importer}.

L'importation peut durer plusieurs minutes. Une fois terminée, la page revient automatiquement sur l'accueil.

\subsection{Affectation des classes}
Pour que la génération des statistiques soit plus précise, chaque classe doit être rattachée à un niveau et une section. Une correspondance est calculée automatiquement à l'importation ; il faut la vérifier et la compléter en allant dans l'onglet \bouton{Options}.

Voilà ! Legion est prêt :)

\subsection{Importation des résultats de fin d'année}
Le chef d'établissement peut obtenir en fin d'année scolaire les listes des élèves ayant passer un examen précisant leur résultat. Ils se présentent sous la forme de fichier xls ; un par filière et par groupe d'examen.

Ces fichiers sont importables directement dans Legion. Aller dans \bouton{Options} et ajouter successivement chaque fichier. La colonne « Diplômé » de la partie \bouton{Liste} sera automatiquement mise à jour.

\end{document}
